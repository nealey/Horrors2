\chapter{The Journey}
\by{January}


The name's Luke Bavarious, private detective. I've seen
some gruesome things in my time. Enough to make a man vomit blood.
That's why I carry a loaded Beretta. Ready to deal expedient
death to a sucker that needs it, or any misshapen foe. But one
morning in 1991, I stumbled into a tragedy that wouldn't be
brought to such an easy conclusion.



It was a seemingly ordinary day. I turned on the TV as I ate my
breakfast. I usually checked the news for the violent crime du
jour, but I wasn't in the mood. I left the television dial to
linger on a children's program, an animated story called
``The Journey''.



A young man decided to go on an expedition to a foreign land. He
selected a group of friends and relatives to join him. The young
visionary's face shone with pride as the preparations began.
Loved ones provided plenty of supplies and all the financial things
for the trip. A celebration was held when the group was ready to
set out.



But some time into the journey, misery befell the adventuring
party. Everyone developed a horrid sickness, the likes of which
none had ever seen. Their eyes sunk into their heads as their
frames grew gaunt and skeletal. Still, they pushed on. It was too
far to turn back.



As they trudged onward, their skin thinned and the color diminished
to putrid green. Pustules developed, swelled, and exploded like
liquid landmines, coating them in moist blankets of rust colored
blood. In the end, every one of them drank of the bitter mercy of
death, as they were reduced to nothing but fetid corpses.



When the story came to its revolting conclusion, I vomited a
fountain of spew, transforming my breakfast cereal into a
despicable acidic cocktail. I couldn't explain the severity
of my reaction. But what were they airing on TV? This looked like a
chapter from the work of a deviant mind --- a day in the life
of Luke Bavarious, perhaps --- not a children's
show.



I grabbed a Coors to soothe my throbbing nerves before work. I was
already late. As I drove, I started to question whether the events
of the morning had really happened. Maybe it had been a
dream.



When I drove past City Hall, I was surprised to see a large
gathering. Something told me I needed to investigate this instead
of continuing to my office. I pushed my way through the crowd to
enter the doors. All around, the atmosphere was one of revelry. A
young man was giving a speech. Banners waved, and well-wishers
cheered.



It was the same man from the story I had just seen! My mouth
dropped open like a gaping black hole as I pondered his cruel fate.
Immediately my veins pulsed and pounded, popping instinctually out
of my neck!



I noticed one young lady whose silence was telling. Far removed
from the merriment, she seemed as out of place as I. Tears trickled
from her bloodshot eyes. I had to I ask. ``Who is that young
man?''



``He's my brother,'' she said.



``He's going to die and take others with him!'' I
exclaimed. ``His plan is foolishness! We must stop
him!''



She did not respond. Her expression was of resignation.



``I must act if no one else will,'' I thought.
``Better one bloody mess than many.'' I drew my Beretta
and aimed it at the young man to make the fatal shot. At the sight
of my weapon, the sister heaved violently. Vomitus sprayed all over
my pants and on my Beretta. I hesitated.



``Don't,'' the girl sobbed. ``I already tried
to convince them not to go, but no one will listen. If you kill my
brother, they'll probably go anyway. We just have to let it
happen.''



I felt the questions frozen in my mind like impending doom.
``How do you know this? How do you know they will
die?''



Tears cascading down her pale cheeks, she looked me in the eye. I
knew the true meaning of hopelessness when she
replied{\ldots}



``I saw it on TV.'' 
 






