\chapauth{Syphilicious!}
\chapter[What Lurks Behind Our Eyes]{What Lurks Behind Our Eyes: The Horrid Reflection Revisited}


Thursday night, and everything is quiet. Unusual for me, but in my
current settings it should be expected; instead of walking my beat in
the thug-infested alleys of our dear city, I am far out in the country,
at Old Woman McCannshire's place, engaged in a staring contest with the
termites that crawl in and out of the floor of her porch as I wait for
her to answer the door. The middle of nowhere does not properly describe
my location; I'd been driving so long that I'm probably already halfway
out. My name is Luke Bavarius, and I'm a detective, but tonight I appear
to be the guy that drives around checking under old biddies' beds for
monsters.

Even the pranks get men sent out these days. A prank is what I would
have thought this would be if I didn't know the old woman calling was
too addled to even have a teenager's sense of humor. McCannshire thinks
her house is haunted by spirits, and wants one of us ``wonderful young
men you have working down there'' to come check it out. I'm almost glad I
forgot to bring my spare ammunition for my Beretta out here; I've used
that thing enough today considering my nerves are just about as shot as
those three bank robbers, and if this goose chase got any more boring
I'd probably put it in my mouth and make brain gumbo.

The unlatching of bolts awakens me from my reverie, and my head snaps
back up into the proper position. ``You win this time, termites,'' I
mutter, wiping a thin string of drool from my chin. Slowly, the door
creaks open, and I am treated to the sight of Mrs. McCannshire in a
wispy white nightgown. Perhaps in the prime of her youth this might have
been something I could have tolerated or even enjoyed, but the broad has
long been in her more tender years of age, her face has more wrinkles
than the wandering Jew's underwear, and her nightgown is greasy with the
mysterious secretions of the elderly. I try to focus on the mangy grey
poodle she cradles in one arm, a dirty little mutt that she probably
pampers like nobody's business. She really fits the picture of an old
bag of bones, and as soon as she opens her mouth I can tell how far gone
she really she is.

``Are you the detective Officer Dent sent over to help with the spirits
in my house?'' She speaks slowly and clearly, her eyes twin moons of
gawkish innocence. I don't know which kind of dementia would be worse:
the flavor Mrs. McCannshire possesses where one is magically returned to
the age of nine or the other one where you think the walls are talking
to you. Although, considering why I was here, it's possible she suffered
from the latter too.

``Uh\ldots yes. Yes, ma'am. Officer Dent is my, uh, superior.'' I step past
her and walk inside, trying to ignore the subdued growl the mutt in her
hands has started up upon sight of me. The place is clean to a point;
there are numerous tables and shelves bedecked with pictures and family
heirlooms, all meticulously dusted, but the carpet is smeared with dirty
pawprints and general dust and filth, it's frayed and ragged material
likely not blessed by the gentle touch of a vaccuum cleaner for
years. The carpet and walls are an ugly matching beige and all the
miscellaneous objects, despite constant care, have lost their
luster. The only sign of real color comes from the bathroom behind the
door opposite the one I had come in, wherein an even more hideous bright
lime green covers the small amount of wall I can see around the door.

I turn to face her, reaching into the folds of my trenchcoat and drawing
out a pack of cigarettes and my lighter. ``Now, what seems to be the
problem here?'' A lazy puff of smoke floats serenely past my raised
eyebrow from my now lit cigarette.

``Well,'' she says, setting the dog down onto the carpet where it does an
annoying little dance around our legs, barking and whining, ``I've been
noticing things for several days now, but only this morning did it get
really bad. You see, every time I use the bathroom I feel someone is
watching me.''

``How can you tell?''

``Well, at first it was just an uneasy feeling. But then I started
hearing voices that would say things that I couldn't make out. Then I
started seeing faces out of the corner of my eye or in a reflection. And
this is happening quite often, mind you. It's happened every time I go
in there, and these days I tend to\ldots oh, how should I say it\ldots
do my business more often, mostly because my--''

``I understand, I understand,'' I say hurriedly. ``Please, continue.''

``Well, uh, this morning, I saw a face in the mirror behind me. And I
didn't just see it, either; it was directly behind me, an entire person,
and he didn't go away until I turned round.''

My eyebrow, having just started to head home for the day, turned
right back around and marched up my forehead. This sounded legitimately
interesting. Whatever had actually happened, seeing a person plain as
day is a lot better than imaginary sounds or tricks of light that even
happened to people who weren't sitting outside Death's doorstep in
motorized wheelchairs. There is really only one thing to do.

``Well, I guess you'll have to show me the bathroom then,
Mrs. McCannshire.''

``Right you are, dear.'' She seems to notice that my gaze has strayed to
the pictures on the small table next to the front door, and as she
hobbles past me towards the bathroom she begins to talk about her dead
husband. Half listening to her talk about the dangers of late term
prostate cancer and wincing at the intimate descriptions she gives of
the times she went with him for his checkups, I search for an ashtray
and find one nestled in between boxes of tissue and stack of gardening
books. I rub the flame out and leave the stub, resolving not to smoke
any more until I leave the house. The old woman doesn't need all that
smoke.

As I join her in the bathroom, I see that her poodle has the same
idea. It flies past me and sits whining at her feet until she relents
and picks it up again. I stand next to her and look around the room. The
mirror is old but clean, and the porcelain throne in the corner is the
same. I look into the sink, and from the short, curly gray hairs lining
the rim I deduce that she washes the dog in it; either that or she's
more up on the trends of women of today than you'd think of a gal her
age.

The horror of the thought further distracts me, and I begin to develop
that thousand yard stare as she tells me about the various scary
encounters she has experienced while voiding her bowels, unnecessarily
clueing me in on the second part in her stories too. Technically I am
looking at the hot water handle, but I am miles away, back on a real
cop's beat or in the arms of a good woman, whichever one does a better
job of distracting me from her current tale of a mysterious voice
whispering in what she thinks is Latin and the effects of the creamed
corn she had with lunch two days ago. Suddenly I spy in the reflection
from the mirror that the dog has the same idea. The yappy little thing
now sits silent and unmoving in her arms, staring intently into the eyes
of its reflection.

At first I am grateful for the relative silence that its new object of
interest has provided, but after a minute it begins to make my skin go
all goosey. I've never seen a dog sit that still for anything. I slowly
move my hand in front of its face, nodding to show Mrs. McCannshire I am
listening at a pause in her latest story involving the cupboard swinging
open and almost hitting her in the head and how the fright really helped
``loosen things, down there''. I pass my hand back in forth in front of
the dog's vision to no effect. In a moment of clarity I drudge up the
dog's name out of its owner's ramblings.

``Jasper! Hey, Jasper!'' At once the dog is a flurry of motion, leaping
out of her hands and latching onto the watch around my wrist with its
teeth. I stumble backwards into the main room and fall to the floor,
frantically batting at the hideous ball of fur as it growls like a
recently castrated bear. Instinct takes over; my mind recognizes when I
am in a fight for my life even when the opponent is a 15-pound owl
pellet. Without thinking I wrap the palm of the hand it grips around its
head and bash it repeatedly against the edge of a bookshelf next to me,
then stagger to my feet and swing it around the room, screaming to match
its rabid cries. All of a sudden it flies free with a high pitched yelp
and collides with the table on which the ashtray rested and the table
and its contents tumble to the ground.

I approach cautiously, waiting for my opponent to make some sign of
life. At once the small pile of picture frames and knicknacks erupts as
Jasper flies straight towards my face.

I have anticipated it; it passes fruitlessly over my head as I lean
backwards almost parallel to the floor, and I hear its frenzied growling
suddenly muffled. I push my spine back into place with one hand and spin
around only to see Jasper hanging from the ledge of a desk, his jaw
wrapped around it and his teeth grinding into it as if he imagined it to
be my arm. I act quickly, sparing no mercy. With several steps I come
upon the helpless creature and I lift a booted foot to hover a foot away
from the back of its skull.

``Chew on this, pooch.''

There is a loud, wet crack as its skull explodes like a balloon filled
with bones and blood. It's corpse falls silently to the floor, followed
by the lower half of his jaw and head. The top half rests on top of the
desk, firmly embedded into the wood. I curse silently to myself and wipe
my foot off on the carpet, leaving behind a red smear flecked with hair
and bits of bone.

All at once I come to my senses, and I turn to see Mrs. McCannshire
standing at the bathroom door. For a second we both stand staring
wordlessly at each other, then she utters a soft cry and flees back into
the bathroom. I hear a soft click as she locks the door behind her.

I sigh and walk over, knocking on the door. ``Mrs. McCannshire, I'm sorry
about Jasper, okay? I shouldn't have\ldots{} done that, but he was, I mean he
was attacking me. There was nothing else I could do.''

I continued to apologize while I listened to her sobs, trying to look
anywhere but back at that head, or that part of it, those sightless eyes
silently judging me. I've killed people before in my line of work, and I
see their faces when I close my eyes, but now this mutt was getting to
me more than any of them ever did. It was an irritable little thing, but
why did it up and attack me like that? What did it see in that mirror?

I notice that the crying on the other side of the door has stopped, and
for a moment I feel relief. ``Mrs. McCannshire, if you can just come out
here we can talk about this. Again, I'm sorry about your dog, but--''

I am interrupted by the click of the lock, and as the door slowly comes
ajar I help her open it. She stands there, head down, and she looks so
depressed that I can't help but resume my apologies. ``If there's
anything I can do to pay you back for what I did, you name it. I really
can't tell you how sorry I am, I'll get you a new dog, whatever you
want. I'm sure I\ldots{}''

The look in her eyes when she raises her head is different than what
you'd think a hysterical old woman would have. They're more intelligent
than they were before, those eyes, and they seem to possess more menace
than I assume an old lady like that would be able to muster.

A bony hand wraps around my throat with otherworldy strength, choking
off the rest of the sentence. She lifts me off my feet, pulls back, and
for a brief moment everything is serene.

Then I hit the wall. I slide down next to the open front door, and after
my eyes uncross and the black in front of my eyes goes away I use the
knob to pull myself up. I check for broken bones and don't find good
news in the ribs area, but other than that I am fine, if bruised.

``Well, you've got a good arm, I have to give you that.'' I think over my
options, running my tongue over my teeth. I can't hurt her; she's
obviously just possessed by whateve possessed that dog in the mirror. I
have to get the spirit out of her, or incapacitate her, but I don't know
how to perform exorcisms and at her age a gust of wind could kill
her. Although if she's able to throw like that maybe she's a lot
stronger in other ways too. What if I tied her up?

Something makes my train of thought come to a screeching halt. It hasn't
reached the station, it's gone straight off the tracks. There were no
survivors.

My brain is recieving messages my tongue shouldn't be sending. It's not
finding something that should be there. I grab a polished silver cup off
a table and flash my teeth at my reflection. There's a black square
where there should be a nice little white one.

I've lost a tooth.

This bitch is going to die.

I toss the cup and pull my piece, my finger already on the
trigger. Worse men talk about how their guns sing songs that only ever
have a few notes; that's played out, and anyway my Beretta never saw the
appeal in singing. It yells, and it only ever needs to raise its voice
once to win an argument with someone.

As I aim down the sights at the old girl now barrelling towards me from
accross the room with a horrifying screech, I recall something about not
having ammunition, and I anticipate the empty little click. Cursing
wildly, I hurl the gun at her, and it bounces off her forehead
ineffectively. I reach for the knife strapped to my leg down at my
ankle, but it is too late; she knocks it out of my hand with one swift
strike just as I am bringing it up and it clatters against the wall. She
slams me up against the same patch of wall that I'd said hello to twenty
seconds ago and holds me at arm's length against the wall, my head more
than two feet higher than hers and my feet off the ground clattering
against the wall. Both hands are wrapped around my neck and I am rapidly
losing oxygen. You need to do something now, I think. Or you're done,
Luke. You're done.

Frantically my hands search for something, anything, to fight her off
with, finding nothing. I'm simply too far off the ground to reach
anything. I turn my head as much as her steel fingers allow, and through
my darkening vision I can barely see an umbrella stand with one large
black umbrella in it. In vain I stretch my left hand towards the handle,
my fingers finding air and then brushing the handle. I strain as hard as
I can as the pain advances and my sight blackens, and suddenly I have a
grip, I grasp it with the very tips of my fingers, bring it up to my
hand. She is laughing now, piercing and mocking, delighting in her
triumph. She doesn't keep it up for long. I raise the umbrella high
above my head then stab it down into her open mouth and throat, pushing
it into her esophagus as she spits and gurgles, her hands clutching even
tighter at my neck. The handle is just past her teeth, my hand gripping
it firmly even as she bites into my wrist. I use my thumb to find the
release and push it up.

The umbrella is spring operated, the fabric edged with sharp metal. Her
neck evaporates in a cloud of blood and her head shoots up into the
hair, twirling in the air like a basketball and falling to the ground
with I and the rest of her body.

After a while, coughing and wheezing, I push her corpse off of me and
use the blood-soaked umbrella to stand up. As soon as I try to walk
towards the nearest chair, I stumble and trip over her head. Standing up
again, I look back down at the bloody mess on the carpet and on me. I
feel bile rising in my throat, and I turn to run to the bathroom.

I push past the door and stagger to the sink, where I vomit noisily and
stand for a while, staring into this puddle of my own sick. After what
seems like forever I look up and into my reflection in the mirror. I am
hunched over the sink, my hands still grasping the sides, my mouth
hanging open and a thin trail of vomit hanging from my lower lip. My
eyes are wet with tears from the choking and the vomiting.

Truly I am a pitiful sight. I give myself a weak smile, as if it will
cheer me up. I can't help but notice that something is off in my
reflection, but I can't think what. Then I tongue the gap where my tooth
used to be. My reflection does not. It still has the full set.

The reflection straightens its back and wipes the vomit away, dries its
eyes with the sleeve of its shirt, and all I can do is stare in dumb
incomprehension. It is the same short black hair, the same baby blue
eyes, the same trenchcoat, the same man, yet it moves of its own free
will. It is me and yet it is not me.

It has an almost condecending look in its eyes as it reaches down below
the sink, to its ankle. It comes back up, my knife in its hands, its
knife, and I cannot move a muscle.

There is a flash of metal. He cuts through my throat like
cheesecake. The arterial spray gives a good portion of the shitty green
paint job a new coat from the opposite side of the color wheel. There is
a brief sense of motion, and I taste ceramic, my body thudding to the
bathroom floor. I move my mouth wordlessly as red begins to creep along
the grout in between the white tiles. I hear a shuffle of fabic as my
other self steps through the mirror and lowers himself from the sink to
the floor. He steps over my body, taking care to not step in the
advancing pool of blood.

My vision begins to cloud for the last time as he casts the knife
absentmindedly down in front me. It slides to a halt next to my
forehead. He begins to walk towards the front door, then stops, turns
around. He walks cooly back to me, crouches in front of me, grimacing at
the blood that is in danger of soiling the knee of his pants. He looks
me in the eyes, and begins to say something, then thinks better of
it. He does nothing for a second, simply watches me dying, then reaches
over, placing an index and middle finger on my eyelids, and then he
slides them shut.

``Good night, Luke.'' 

